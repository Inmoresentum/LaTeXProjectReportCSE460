\lstdefinestyle{verilogStyle}{
    language=Verilog,
    basicstyle=\small\ttfamily,
    keywordstyle=\color{blue},
    commentstyle=\color{green},
    stringstyle=\color{purple},
    showstringspaces=false,
    breaklines=true,
    numbers=left,
    numberstyle=\tiny\color{gray},
    stepnumber=1,
    numbersep=4pt,
    frame=single,
    framerule=0pt,
    framesep=0pt,
    framexleftmargin=5pt,
    xleftmargin=10pt,
    tabsize=4,
    captionpos=b,
    floatplacement=tbhp
}

\subsection{Verilog HDL Code}\label{subsec:verilog-hdl-code}
\begin{lstlisting}[style=verilogStyle, caption={Verilog code for 4-bit ALU},label={lst:lstlisting}]
module project(input clk, input [3:0] A, input [3:0] B,
 input [2:0] opcode, output reg [3:0] C,
  output reg carr, output reg sign, output reg zero);
// Will be using state to indicate state of the machine.
reg [1:0] state = 0;
always @ (posedge clk) begin
    case (state)
        2'b00: begin
            case (opcode)
                3'b000: begin
                    C <= 4'b0000; //RESET operation
                    carr <= 1'b0;
                    sign <= 1'b0;
                    zero <= 1'b1;
                end
                3'b001: begin //XNOR operation on LSBs
                    C[0] <= ~(A[0] ^ B[0]);
                    zero <= C[0] == 1'b0;
                end
                3'b010: begin //SUB operation on LSBs
                    {carr, C[0]} <= B[0] - A[0];
                    zero <= C[0] == 1'b0;
               end
                3'b011: begin //NAND operation on LSBs
                    C[0] <= ~(A[0] & B[0]);
                    zero <= C[0] == 1'b0;
                end
                3'b100: begin //ADD operation on LSBs
                    {carr, C[0]} <= A[0] + B[0];
                    zero <= C[0] == 1'b0;
                end


            endcase
            state <= 2'b01;
        end
        2'b01: begin
            case (opcode)
                3'b001: begin //XNOR operation on next bit
                    C[1] <= ~(A[1] ^ B[1]);
                    zero <= zero & (C[1] == 1'b0);
                end
                3'b010: begin //SUB operation on next bit
                    {carr, C[1]} <= B[1] - A[1] - carr;
                    zero <= zero & (C[1] == 1'b0);
                end
                3'b011: begin //NAND operation on next bit
                    C[1] <= ~(A[1] & B[1]);
                    zero <= zero & (C[1] == 1'b0);
                end
                3'b100: begin //ADD operation on next bit
                    {carr, C[1]} <= A[1] + B[1] + carr;
                    zero <= zero & (C[1] == 1'b0);
                end


            endcase
            state <= 2'b10;
        end
        2'b10: begin
            case (opcode)
                3'b001: begin //XNOR operation on next bit
                    C[2] <= ~(A[2] ^ B[2]);
                    zero <= zero & (C[2] == 1'b0);
                end
                3'b010: begin //SUB operation on next bit
                    {carr, C[2]} <= B[2] - A[2] - (carr & ~zero);
					zero <= zero & (C[2] == 1'b0);
                end
                3'b011: begin //NAND operation on next bit
                    C[2] <= ~(A[2] & B[2]);
                    zero <= zero & (C[2] == 1'b0);
                end
                3'b100: begin //ADD operation on next bit
                    {carr, C[2]} <= A[2] + B[2] + carr;
                    zero <= zero & (C[2] == 1'b0);
                end

            endcase
            state <= 2'b11;
        end
        2'b11: begin
            case (opcode)
                3'b001: begin //XNOR operation on MSBs
                    C[3] <= ~(A[3] ^ B[3]);
                    sign <= C[3];
                    zero <= C == 4'b0000;
                end
                3'b010: begin //SUB operation on MSBs
                 {carr, C[3]} <= B[3] - A[3] - carr;
                 sign <= C[3];
                 zero <= C == 4'b0000;
                   if (B[3] < A[3]) begin //if result is negative, take two's complement of result
                       C <= ~C + 4'b0001;
                       sign <= C[3];
                   end
                end
                3'b011: begin //NAND operation on MSBs
                                    C[3] <= ~(A[3] & B[3]);
                    sign <= C[3];
                    zero <= C == 4'b0000;
                end
                3'b100: begin //ADD operation on MSBs
                    {carr, C[3]} <= A[3] + B[3] + carr;
                    sign <= C[3];
                    zero <= C == 4'b0000;
                end

            endcase
            state <= 2'b00;
        end
    endcase
end
endmodule
\end{lstlisting}
